% Options for packages loaded elsewhere
\PassOptionsToPackage{unicode}{hyperref}
\PassOptionsToPackage{hyphens}{url}
%
\documentclass[
]{article}
\title{Redistribution and Place-Based Economies}
\author{Noah Daponte-Smith}
\date{4/4/2022}

\usepackage{amsmath,amssymb}
\usepackage{lmodern}
\usepackage{iftex}
\ifPDFTeX
  \usepackage[T1]{fontenc}
  \usepackage[utf8]{inputenc}
  \usepackage{textcomp} % provide euro and other symbols
\else % if luatex or xetex
  \usepackage{unicode-math}
  \defaultfontfeatures{Scale=MatchLowercase}
  \defaultfontfeatures[\rmfamily]{Ligatures=TeX,Scale=1}
\fi
% Use upquote if available, for straight quotes in verbatim environments
\IfFileExists{upquote.sty}{\usepackage{upquote}}{}
\IfFileExists{microtype.sty}{% use microtype if available
  \usepackage[]{microtype}
  \UseMicrotypeSet[protrusion]{basicmath} % disable protrusion for tt fonts
}{}
\makeatletter
\@ifundefined{KOMAClassName}{% if non-KOMA class
  \IfFileExists{parskip.sty}{%
    \usepackage{parskip}
  }{% else
    \setlength{\parindent}{0pt}
    \setlength{\parskip}{6pt plus 2pt minus 1pt}}
}{% if KOMA class
  \KOMAoptions{parskip=half}}
\makeatother
\usepackage{xcolor}
\IfFileExists{xurl.sty}{\usepackage{xurl}}{} % add URL line breaks if available
\IfFileExists{bookmark.sty}{\usepackage{bookmark}}{\usepackage{hyperref}}
\hypersetup{
  pdftitle={Redistribution and Place-Based Economies},
  pdfauthor={Noah Daponte-Smith},
  hidelinks,
  pdfcreator={LaTeX via pandoc}}
\urlstyle{same} % disable monospaced font for URLs
\usepackage[margin=1in]{geometry}
\usepackage{graphicx}
\makeatletter
\def\maxwidth{\ifdim\Gin@nat@width>\linewidth\linewidth\else\Gin@nat@width\fi}
\def\maxheight{\ifdim\Gin@nat@height>\textheight\textheight\else\Gin@nat@height\fi}
\makeatother
% Scale images if necessary, so that they will not overflow the page
% margins by default, and it is still possible to overwrite the defaults
% using explicit options in \includegraphics[width, height, ...]{}
\setkeys{Gin}{width=\maxwidth,height=\maxheight,keepaspectratio}
% Set default figure placement to htbp
\makeatletter
\def\fps@figure{htbp}
\makeatother
\setlength{\emergencystretch}{3em} % prevent overfull lines
\providecommand{\tightlist}{%
  \setlength{\itemsep}{0pt}\setlength{\parskip}{0pt}}
\setcounter{secnumdepth}{-\maxdimen} % remove section numbering
\usepackage{booktabs}
\usepackage{siunitx}
\newcolumntype{d}{S[input-symbols = ()]}
\usepackage{longtable}
\usepackage{array}
\usepackage{multirow}
\usepackage{wrapfig}
\usepackage{float}
\usepackage{colortbl}
\usepackage{pdflscape}
\usepackage{tabu}
\usepackage{threeparttable}
\usepackage{threeparttablex}
\usepackage[normalem]{ulem}
\usepackage{makecell}
\usepackage{xcolor}
\ifLuaTeX
  \usepackage{selnolig}  % disable illegal ligatures
\fi

\begin{document}
\maketitle

\hypertarget{regional-deservingness-and-place-based-economic-resentment}{%
\subsection{Regional deservingness and place-based economic
resentment}\label{regional-deservingness-and-place-based-economic-resentment}}

I first look at the relationship between preferences over redistribution
(my dependent variable) and perceptions of the geographic distribution
of central-government funding in Britain. The key independent variables
are a series of questions that ask respondents about whether whether
various geographic areas receive their ``fair share'' of central
government spending. The scale is from 1-5, from ``much less than its
fair share'' to ``much more than its fair share.'' Respondents were
asked about three geographic areas: their local area, their region, and
London. I plot the distributions of these variables below. Most
respondents believe their local areas and regions are under-funded
relative to what they deserve, while the large majority believe London
is overfunded.

\includegraphics{uk_econ_resentment_files/figure-latex/histograms-1.pdf}

I also plot the geographic distribution of these attitudes. The left and
right panels plot the mean value by region of the regional-fair share
and London-fair shares variables respectively. Scotland, Wales, London,
and South East England are the only regions that rank above the median
value (by region) of the regional fair-share variable. This accords with
the facts that the latter two regions are either London or its immediate
environs, while centralized redistribution to Scotland and Wales'
devolved authorities has high political salience in British politics.
The right panel shows that \textit{every} region, including London
itself, believes London receives at least its proper share of central
government spending, but that this feeling is most pronounced in
northern England. Again, this accords with the prominent trend of
north-south divides in British politics.

\includegraphics{uk_econ_resentment_files/figure-latex/unnamed-chunk-1-1.pdf}

\hypertarget{relationship-with-preferences-over-redistribution}{%
\subsection{Relationship with preferences over
redistribution}\label{relationship-with-preferences-over-redistribution}}

I now look at the relationship between these variables and a political
outcome - preferences over redistriubtion. In eahc model, the key
dependent variable is a 1-10 scale; higher values indicate preferences
for higher levels of redistribution. Models 1 and 2 use the local and
regional measures respectively as independent variables. For models 3
and 4, the dependent variable is the difference between the respondent's
response to the London question and the local area and region questions
respectively. As such, these latter models assess the respondent's
perception that London's share of central government spending exceeds
that of their own local area or region. This might plausibly capture a
sense of place-based distributional resentment. I use linear models in
each case. I include regional fixed effects, alongside a suite of
demographic covariates - age, gender, social grade, education, household
income, and the respondent's left-right placement. Positive coefficients
indicate higher values of the variable correspond with preferences for
higher redistribution. \newpage

\begin{table}

\caption{\label{tab:fairshare_redist}Perceptions of geographic redistribution and preferences over redistribution}
\centering
\begin{tabular}[t]{lcccc}
\toprule
  & Model 1 & Model 2 & Model 3 & Model 4\\
\midrule
Local area gets fair share & \num{-0.226}*** &  &  & \\
 & (\num{0.030}) &  &  & \\
Region gets fair share &  & \num{-0.079}** &  & \\
 &  & (\num{0.029}) &  & \\
London-local fair share &  &  & \num{0.164}*** & \\
 &  &  & (\num{0.020}) & \\
London-region fair share &  &  &  & \num{0.096}***\\
 &  &  &  & (\num{0.022})\\
Education & \num{-0.072}*** & \num{-0.065}** & \num{-0.074}*** & \num{-0.065}**\\
 & (\num{0.021}) & (\num{0.022}) & (\num{0.022}) & (\num{0.024})\\
Age & \num{-0.014}*** & \num{-0.014}*** & \num{-0.013}*** & \num{-0.013}***\\
 & (\num{0.002}) & (\num{0.002}) & (\num{0.002}) & (\num{0.002})\\
Male & \num{-0.089}+ & \num{-0.143}** & \num{-0.079} & \num{-0.131}*\\
 & (\num{0.046}) & (\num{0.049}) & (\num{0.049}) & (\num{0.053})\\
Social grade & \num{0.110}*** & \num{0.113}*** & \num{0.106}*** & \num{0.119}***\\
 & (\num{0.018}) & (\num{0.019}) & (\num{0.019}) & (\num{0.020})\\
Household income & \num{-0.105}*** & \num{-0.107}*** & \num{-0.103}*** & \num{-0.102}***\\
 & (\num{0.007}) & (\num{0.008}) & (\num{0.008}) & (\num{0.009})\\
Left-right self-placement & \num{-0.668}*** & \num{-0.680}*** & \num{-0.660}*** & \num{-0.670}***\\
 & (\num{0.011}) & (\num{0.012}) & (\num{0.012}) & (\num{0.013})\\
\midrule
Num.Obs. & \num{12204} & \num{10624} & \num{10760} & \num{9140}\\
R2 & \num{0.321} & \num{0.320} & \num{0.316} & \num{0.311}\\
R2 Adj. & \num{0.320} & \num{0.318} & \num{0.315} & \num{0.310}\\
Std.Errors & HC2 & HC2 & HC2 & HC2\\
\bottomrule
\multicolumn{5}{l}{\rule{0pt}{1em}Linear models with region fixed effects.
             Dependent variable is preferences over redistribution on 1-10 scale.}\\
\multicolumn{5}{l}{\rule{0pt}{1em}+ p $<$ 0.1, * p $<$ 0.05, ** p $<$ 0.01, *** p $<$ 0.001}\\
\end{tabular}
\end{table}

Let us first look at models 1 and 2. In each model, the key coefficients
are negative and significant: As respondents believe their local area or
region receive increasingly ``fair'' shares of central government
spending, their preferences for generalized redistribution decrease.
Models 3 and 4 look at respondent's perceptions that London receives an
undue amount of central spending - this may capture perceptions of
peripheralness, or perhaps place-based economic resentment. In both
models, the coefficients are positive and significant: Respondents who
believe London receives too much spending are also more favorable
towards redistribution. Across all four models, the coefficients in the
local-area models (1 and 3) are larger than in the regional models (2
and 4); this suggests that local areas may be \textit{more} salient to
voters than are regions. The broad takeaway from these models, as I see
it, is that perceptions of deservingness do correspond with preferences
over redistribution.

Next steps:

\begin{enumerate}
\item Look at the relationship between the fair-share variables and objective conditions. Do respondents correctly perceive that their areas receive less than they should (however that may be defined) from the central government? This will depend on the availability of official statistics on distributional formulas. 

\item Investigate the relationship with both economic perceptions and objective economic conditions. When the local economy begins to suffer, do individuals come to perceive their areas as receiving less than their fair shares of funding? 


\end{enumerate}

\end{document}
